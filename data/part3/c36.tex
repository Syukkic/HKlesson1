\hyperref[sec:sec17]{\section{一國兩制還有將來嗎?}}
\label{sec:sec36}

在回答一國兩制有沒有將來之前,得先回答另一個問題:一國兩制有沒有期限?坊間常有誤會,以為一國兩制到了特區成立第五十年,即是二零四七年的時候便會自動撤銷。從法律條文來看,這並非事實。從政治現實來看,也不見得二零四七年是必然的分界線。

這個誤解通常源於兩處。第一,是九七前中國政府對香港有「五十年不變」的承諾,帶來對港政策會否於二零四七年完全改變的擔憂。這說法的具體呈現,是《基本法》第五條「香港特別行政區不實行社會主義制度和政策,保持原有的資本主義制度和生活方式,五十年不變」的條文。不少輿論見到此條文,便進一步肯定二零四七年便是特區完結之時。對此,法律界認為這是對條文的錯解,《基本法》沒有「自動過期」的設定,憲制上香港特區和《基本法》在二零四七年以後仍可繼續存在。

第二,是香港在九七前有大批按照《中英聯合聲明》附件三批出的土地契約,列明將會在二零四七年六月三十日到期,考慮到「九七問題」在一九七零年代被英方提出,正正是沿於土地契約期限的問題,也讓不少香港人以為特區政府沒有能力承諾二零四七年之後的事情,進而得出特區將於二零四七年自動撤銷的推論。現實是特區政府不單止有權承諾二零四七年之後的事情,而且已經批出多項時限超越二零四七年的土地契約,例如香港廸士尼樂園的土地條款就是寫到二一零零年的。至於現有將於二零四七年到期的土地契約,預期特區政府會在適當的時候提出續期機制。

話雖如此,憲制上二零四七年對香港仍有技術上的意義。《基本法》第一百五十九條規定其修改「不得同中華人民共和國對香港既定的基本方針政策相抵觸」,而《中英聯合聲明》第三條第十二款規定中國對香港的基本方針政策「五十年內不變」。那麼,理論上到了二零四七年,《基本法》的修改就再沒有限制,便可以重頭到尾改得面目全非,這也可以被理解為「特區終結的一天」。不過這件事不會自動發生,最起碼還要符合第一百五十九條的其他規定,例如立法會全體議員三分之二多數同意。當然,如果民主派在立法會的議席數目進一步下降至少於三分之一,這個門檻將會失去實際意義。

對於很多香港人來說,只要有改變的可能,就代表有機會變得更差,而這也是為何當初「五十年不變」的說法在九七前對香港人有吸引力。回頭去看,「五十年不變」的承諾可能一開始就不是一個好主意,因為港英政府的制度本身也有問題,而這些制度被特區政府挪用後更帶來了更多問題(見問題二十)。相對於凝固時空,積極一點去想,二零四七年時否也是一個拆牆鬆綁的機會,重構香港與中國的關係,為香港建立一個更能保障自主的政治安排?

因此,不少年輕人把二零四七年理解為「二次前途問題」。到了二零四七年,現在的少年人正值壯年,香港的政治地位屆時可否改變對他們來說並不是一個抽象的問題,而關係到他們應否把自己的青春投放在香港。有些年輕人就提議要為二零四七年後的香港地位舉行公投,並認為屆時讓香港獨立可以是一個選項,也就是所謂的「自決」。這種想法的基礎,是雖然《基本法》第一條列明香港是中國的一部分,但《基本法》是可以修改的,只要不違反中國對香港的「基本方針政策」就可以;而既然「香港成為一個特區」這個「基本方針政策」到二零四七年就會到期,那麼邏輯上要求中國政府在二零四七年容許香港獨立,條文上是不應被視為違反《基本法》和《中英聯合聲明》的。現實上中國政府屆時會否容許是一回事,但最起碼邏輯上說得過去。可以想像,香港政府並不同意這種理解。事實上,香港政府已經把認同這種理解視為不服從《基本法》,並以此為理據拒絕持此政見者參選立法會。

話雖如此,就算香港獨立的選項真的存在,又有多少香港人會支持呢?按香港中文大學傳播與民意調查中心於二零一七年所作的調查,只有11.3\%的受訪者支持香港於二零四七年後獨立,表示反對的卻有60.2\%,可見港獨即使作為一個假設選頂也沒有多大的支持。支持屆時繼續一國兩制的,則有71.2\%,明顯是社會主流。不過,這並不代表民意認為現時香港的情況十分理想。相反,在同一調查當中,有62.9\%的受訪者認為「香港回歸後整體社會狀況」變差。如是者,港獨目前不受歡迎,並不代表香港人很滿意現時的情況,也可能是他們覺得情況未差到要追求港獨,或支持港獨者未能提供一個更吸引的說法。另一方面,即使多數人認同繼續一國兩制,《基本法》能否作出一些根本的修改以更有效保障香港自主,公眾討論也十分有限。

從學理上看,目前港獨的論述基礎尚未成形。按學者王慧麟的分析,無論以天然權利或補救性分離權利出發,港獨的法理基礎仍然薄弱(至於要求回歸英國的說法,更是直接忽視了英國早已於1985年通過的《香港法》當中確立了於一九九七年七月一日終止對港主權,並不存在《南京條約》繼續有效的可能,更別說不承認不平等條約早已是國際法學界共識)。相對來說,王慧麟認為循聯合國《公民權利和政治權利國際公約》以及《經濟、社會、文化權利的國際公約》去強調香港人的內部自決權利,更有基礎。

當然,上面只是從理念層面探討港獨。現實上,現代歷史中邊緣地區能脫離一個專制軍事大國成功獨立,只有一九九一年前蘇聯解體一個案例。但把這個案例拿到香港卻有兩個問題:第一,當時蘇聯經濟崩潰而又正經歷熱切的民主浪潮,這兩點在當前的中國都不識用;第二,即使這些邊緣地區成功獨立,當俄羅斯重新走向專制後,不見得它們可以完全脫離俄羅斯的影響。如是者,在目前中國的經濟和政治形勢下,港獨並不是一個現實的選項。

雖然港獨的支持度有限,但容許討論港獨的支持則明顯廣泛,後面涉及言論自由作為香港核心價值的普遍認同。回到二零零三年就《基本法》第二十三條立法的討論,當時輿論普遍認為只要不涉及組織煽動暴力行為,則所謂的港獨活動不應受到規管。

值得注意的是建制陣營比非建制陣營更喜歡談及港獨。畢竟,在當前的政治結構下他們有很大誘因藉炒作港獨議題來向中央政府表達忠誠。梁振英就經常被謔稱為「港獨之父」,因為他十分熱衷於批判港獨,而在他上任行政長官前主流輿論對港獨討論本來並不關注。例如當他在二零一五年的施政報告中點名批評由香港大學學生會出版的《香港民族論》後,該書隨即大賣特賣,廣受各界關注熱議,如同獲得免費廣告宣傳一樣。中國大陸媒體對於港獨議題也有明顯的政治操作,往往會把一些明顯不相關的香港政治爭議冠上港獨之名,使這些爭議在中國大陸民眾的眼中失去正當性,免得他們同情香港人的訴求。

回到二零四七年對香港的意義,從中央政府的觀點出發,理論上可以是一個進一步收緊對香港控制的機會。除了前文提及的修改《基本法》外,更極端的做法是直接把它直接廢除,畢竟它的實施在法理上僅源於一紙命令,待《中英聯合聲明》所載的保證過期後,也可經由一紙命令撤銷。當然,除非屆時國際關係和今天的極為不同,如果中國政府真的以這種方式在二零四七年改變香港的地位,將會對香港的繁榮和穩定帶來不可估量的打擊。

相對來說,對中方較為有利的做法,是在現有《基本法》的框架下以各種手段架空香港的行政權、立法權和司法權,使得港人治港變得有形無實,也同樣可帶來收緊對香港控制的效果。回到一九八四年,《中英聯合聲明》尚在草議,時任行政局首席非官守議員鍾士元與時任國家領導人鄧小平會面,曾當面提出三項憂慮:第一,是特區表面上港人治港,但治港的港人都由北京控制;第二,是處理中港關係的低級幹部不能落實中央政策,干預香港自治;第三,是將來出現走極左路線的國家領導人,改變國策,使得一國兩制無法有效實行。回顧特區成立以來的種種政治爭議,第一項憂慮早已成真,而第二和第三項在過去數年也變得十分明顯。換言之,中國政府要從實際上改變一國兩制,其實不用等到二零四七年已可發生,或已經發生。

那麼一國兩制是否就沒有未來?讓我們回到起點:一國兩制的初衷,從香港的角度出發,一言蔽之就是中國大陸的制度和香港的不一樣,而且不適合香港。那麼兩套制度有什麼分別?過去的說法,是香港實施資本主義而中國大陸實施社會主義。但無論從學理上或事實上去討論,來到今天這已不是一個好的分類。最起碼,今天中國的經濟結構已和一國兩制最初提出的時候很不一樣。三十多年來沒有改變的,是香港人對自由的追求,和中國大陸政治上的封閉。中港兩套制度在對待自由的基本分別,才是當初中國需要向香港承諾兩制的根本原因。然而,沒有自由的土壤,政權的承諾又可由誰來監督?於是乎,一國兩制從一開始就存有本質上的矛盾:如果中國的政治制度是開放的,兩制其實並非必須;但當中國的政治制度是封閉的,兩制就算設立了也沒有現實保障。回到《中英聯合聲明》訂立的時候,中國大陸剛剛走出文革陰影,向世界學習和接軌是中國社會的主流想法,一國兩制本質上的矛盾也就暫且被忽略。今天的中國已不是三十多年前的中國,矛盾也就徹底曝露了。

說到這兒,可見無論香港人喜歡或願意與否,一國兩制的未來和中國政治的前景難以分割。近年中國政治明顯變得更為封閉,經濟方面的「國進民退」被批評為違反改革開放的初衷,社會方面曾一度有所發展的公民社會近年受到嚴重打壓。與此同時,中國的外交策略也捨棄了過去的韜光養晦,「亮劍」等浮誇之詞變得普遍,世界各國各地人民對於應否接受一個專制政體成為超級大國變得有所警剔。面對這些阻力,中國政府則以民族主義措辭來誤導中國人民忽視問題的源由。放在這宏觀格局中去看,香港政治在近年遇到的種種問題,以及香港人對中國大陸所產生的排拒並非偶然,而是香港處於當代中國變遷與世界碰撞的最前沿位置,衝擊自然來得最為突出。

\rule[-10pt]{15cm}{0.05em}

伸延閱讀:

2013年度香港大學學生會學苑(2014):《香港民族論》,香港:香港大學學生會。

王慧麟(2017):〈失落的自決 自決的失落〉,王慧麟等篇《本土論述2015-2017》,香港:印象文字。

陳智傑、王慧麟(2012):〈香港人的國家認同態度〉,本土論述編輯委員會、新力量網絡編《本土論述 2012》。