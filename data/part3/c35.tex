\section{普選可以解決香港所有問題嗎?}

普選是解決香港政制問題的其中一個必須條件。普選和香港政制就好像水和生命的關係:不是說有水就一定能健康生活,但沒有的話就肯定維持不了。同樣地,普選不足以解決香港所有的政制問題,但沒有普選則無法解決其中一個最關鍵的問題:政府缺乏認授和沒有輪替機會。當然,即使有普選,選舉制度設計和政黨制度的完備,以及整個公民社會的成熟,同樣十分重要。

在現存的政治制度下,無論政府的提議是好是壞,最起碼地區直選產生的立法會議員不會有太多誘因主動支持。他們的選民基礎比行政長官遠遠要闊,其認授基礎更高。行政長官的選民基礎和他們的又基本不重疊,他們不用擔心反對行政長官等同得罪了曾投他一票的選民。因此,特區政府施政不順,其實是整個政治制度本身使然。

香港政制一方面不容許反對者有執政的機會,同時在立法機關又設下重重限制使它失去建設功能,只餘抗議的角色。這個設計在制度上產生了一些極壞的傾向:無論是執政者或反對者都沒有誘因提高其執政和議政能力,也沒有誘因相互合作和建立信任。要理順朝野關係,讓雙方不會把對方迫死,甚至促使互相學習,最直接的方法就是輪替。當在朝的知道自己會有在野的一天,在野的知道自己會有在朝的機會,雙方的行為規範就會改變。不過在香港,在普選遙遙無期的前設下,雙方都心知肚明角色互換的機會甚低。長期的糾纏關係,最終形成一個雙輸局面:弱行政機關對弱立法機關,香港自然停滯不前。而正如前文所述,這個結果很可能是中央政府刻意阻延普選的目的,以強化在香港的影響力。

學者馬嶽認為香港政制內在矛盾的根本,在於中央政府「要說服香港人,非民選的行政機關而非部分民選的立法會,才是公眾利益的真正代表;它要說服香港人,他們有足夠的聰明才智去行使他們的社會和經濟自由,但過於愚蠢去選出自己的領袖;它要說服香港人,即使沒有民主制度也可以保護高度自治,因為中央政府在干預的時候會知所進退。」很明顯,這些要求存有明顯的自相矛盾;也只有普選,香港才有機會走出這些矛盾。

從現實的政策倡議去看,可看到缺乏認授的政府處處舉步維艱。以「強積金對沖」等涉及商界和勞工利益的政策為例,由於選出行政長官的選舉委員會由商界主導,當政府提出一些對商界有利的建議時,輿論便會立即質疑政府是要借機會回報他們的支持,而非真正為社會整體利益著想。這些質疑使得政府在推動各種政策時往往事倍功半。

同樣道理,由於選舉委員會的大多數成員都聽命於中央政府,使得很多市民無法相信行政長官在處理一些中央政府和香港利益可能有衝突的情況時,會認真考慮香港一方的需要。例如《基本法》第廿三條的立法問題,由於涉及國家安全和個人自由之間的取捨,就有政黨認為如果政府沒有普選的制約,可以容易以國家安全為藉口利用相關法規打壓政敵,因此提出只有在政府實施普選後才可推行立法工作。

解決香港的政治困局,普選是第一步,而這一步早就應該要踏出。特區成立二十二年來,香港不單早已經準備好實施普選,而且有多項改革都因沒有普選而難以推行,已是各界共識。過去曾有輿論認為普選雖然重要,卻又聲稱香港不適合或還未準備好普選,並引用外國民主化後未能鞏固引發社會混亂的案例為由,拖延普選進程。對此,本地政治學者已有不少專著反駁,強調香港恐怕是世界上眾多尚未全面民主化的地方當中,準備最為充份和條件最為成熟的地方。

以經濟因素為例,政治學者對於民主發展是否需要一定的經濟基礎,曾有很多爭論。現時的主流意見認為兩者沒有必然的因果關係(經濟好的地方不一定會追求民主,經濟差的地方一樣有可能鞏固民主),不過經濟發展水平高的地方因為社會不穩的因素比較少,民主制度在設立後較易持續。考慮到香港的人均生產總值位處世界前列,經濟基礎的問題對香港並不重要。也有學者認為一個社會的經濟基礎廣闊,例如民眾不是都服務同一個顧主的話,帶來的生計自由也有利民主鞏固。對此,香港奉行市場經濟,政府對經濟的干預也相對較少,為民主發展帶來良好基礎。

另一個過去常見的爭議,是所謂的華人社會或儒家思想不利民主(例如強調長幼有序而不是人人平等,強調群體利益而不是個人權利)。此等說法有兩個主要問題:其一是無視了文化內涵的多樣性(例如儒家思想強調「選賢與能」,也可理解為和民主原則相呼應);其二是無視了文化本身不斷改變,文化也可以是政治制度的產物。最後,同受儒家思想影響深遠的台灣社會已發展出穩定的民主政治,經歷多次政黨輪替和政治危機後仍能維持民主制度,華人社會不能建立民主的說法已不再有說服力。

要在香港實施普選,除了中國政府的因素外,目前最需要的是社會和制度配套。學術研究普遍認為活躍的公民社會是民主鞏固的重要關鍵,社會中的各種非政府組織,如商會、智庫、專業組織、倡議組織,以至街坊互助會等,都是防範政府濫權的重要力量。對此,香港本來已有相當活潑的公民社會,只因政治壓制而未能有所發揮(見問題二十九)。民主政治和公民社會有相輔相成的關係,缺一不可。

同樣和民主政治有共生關係的是管治上的改革。首先,政府要成立獨立專業的機構負責選舉事務,以免當權者利用不公平的選舉來讓自己可以永續當選。議會也要訂立《政黨法》和《政治獻金法》,以規範政黨運作和避免金權政治。而要讓議會和公眾能有效監督政府運作,訂立《檔案法》和《信息公開法》也是必不可少。這些應和普選並行的改革,近年在香港政界已有不少議論,可惜都因為普選本身未能落實而舉步維艱。

到了香港實現普選,並通過上述的各種配套完善各種制度和培育民間力量,香港現時的行政立法僵局才有機會理順,很多拖延已久的公共政策問題,包括房屋、土地、規劃、勞工、醫療、社會保障和教育等,才可以得到認真處理。當香港政治不用再空轉,很多結構性問題都能得到解決。

實現普選,也是處理香港政制終極問題的起點:特區的政治地位問題。

前文提到,香港作為一個自治政體,表面上擁有極高的自治權力,實制的權力卻可能比美國聯邦制下的一個州還要少(見問題十六)。當中的核心問題,是中華人民共和國本身是一個專制政權,而一個專制政權能夠成功做到中央與地方分權,保障地方政府的權益,世上並無先例。當中的關鍵,在於要從制度上保障地方政府的權益,必先有獨立訟裁中央與地方關係的機制。然而在一個專制政權之下,獨立於中央政府的權力機關(如獨立的最高法院),並不可能存在。

放在香港,當中港關係出現矛盾時,現時制度上的處理方法是香港的終審法院向全國人民代表大會尋求解釋《基本法》,而現實上人大常委則經常在終審法院沒有要求的情況下,自行解釋《基本法》。這樣的安排,為香港的政治制度帶來了很多問題,因為人大常委在中港關係當中不可能擔當獨立訟裁的角色(見問題二十五)。這個香港政制的根本問題,即使普選行政長官和普選立法會也不能解決。普選可帶來的改變是屆時的行政長官和立法會可以有足夠的認授,去和中央政府討論這個問題。

至於具體的解決方法,近年有香港輿論提出重訂《基本法》,把當中向中國大陸傾斜的條文拿走。理論上這可以是一個出路,但後面卻帶出一個更根本的問題:《基本法》本身是全國人民代表大會按中國《憲法》第三十一條訂立的。人大代表如何履行他們的權力,重訂《基本法》之後中國政府是否就必定會遵守,不遵守的話會否得到糾正,在現行的中國政治制度之下,全部都沒有保證。

如是者,問題便回到從八十年代「民主回歸論」開始,一個香港政界糾纏不休的爭議:是否中國沒有民主,香港就一定沒有民主?這個說法背後的理據十分簡單:香港受中國管治,如果中國本身沒有民主的話,基於自身政權的考量,也就沒有誘因在香港推動民主,反而有很大的誘因打壓香港的民主,以免香港的民主訴求擴散到中國大陸,撼動中央政權。因此,自八十年代起很長的一段時間內,「中國應當民主化」可以說是香港民主運動的一個基本共識。

不過,這個共識在近年受到明顯挑戰,如在六四紀念晚會是否該繼續喊「建設民主中國」的口號已成為爭議。首先,近年中國政府對中國大陸公民社會的打壓變得激烈,中國民主化的期望對很多香港人來說漸漸變得遙不可及。中國大陸近年的快速經濟增長,表面上也為中共政權帶來強大的表現認授,香港人看不到中共政權短期內會有倒台的可能,甚至懷疑大陸民眾其實十分擁護中共政權。中國政府近年對香港民主運動的打壓也變得強硬,使得一些民主運動的參與者感到支援中國的民主運動已不是他們能力所及的事情。最後,也有意見認為民主的中國不一定會是一個分權的中國,即使中國民主化後也不一定會尊重香港的自主性,甚至擔憂中國的民主化只會帶來多數人的暴政,危害香港自主。

凡此種種,使得香港輿論近年出現了一種新的聲音:如果香港在專制中國之下不可能尋求自主,而民主中國又變得不能期盼,則香港只剩獨立一途。

目前來說,把這種想法付諸實行的,只佔香港社會的極少數。但香港獨立的訴求之所以會出現,代表不少香港人感到一國兩制已無出路,連過去「讓一國兩制重回正軌」的口號也不想再喊,轉而為中港關係的未來尋求新的可能。

\rule[-10pt]{15cm}{0.05em}

伸延閱讀:

馬嶽(2016):〈香港是否「條件成熟」推行民主?〉,馬嶽編:《民主十問》,香港:香港城市大學出版社。

Ma N, 2007, An Institutionalist’s Conclusion, \textit{Political Development in Hong Kong: State, Political Society , and Civil Society}, Hong Kong University Press