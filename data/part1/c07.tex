\section{香港人當年有否喜迎九七?}
\label{sec:sec7}

回到九七年七月一日凌晨零時零刻,當時香港人的普遍心態是期待多於不安。舊立法局雖然有因為無法順利過度而被迫走的議員抗議(見\hyperref[sec:sec6]{問題六}),但有更多的市民在道路兩旁歡迎解放軍入城。離開六四鎮壓後的全城愁雲慘霧只不過是短短八年,香港人在這極短時間內要接受中國管治的現實,當中的社會心理糾結十分值得分析。

上文提到香港人在八九民運當中的參與,在當時的環境下可理解為一種「愛國」的嘗試。隨著時間距離一九九七年越來越近,整個社會也再次重新學習回答「要做一個怎樣的中國人」的問題。當中央政府拒絕「民主愛國」這條進路時,不少香港人便轉用別的方法,特別是政治上相對安全的方法。

這個轉變在演藝界特別明顯。香港演藝界在支持八九民運方面可謂擔當了中堅角色,除了「民主歌聲獻中華」外又創作了歌曲《為自由》,絕大多數的當紅歌星都有參與演唱。兩年後,同一群的歌星加上同一位的指揮,又一同走進錄音室合唱了另一首和中國大陸相關的歌曲,不過這次的理由換成了華東水災。歌曲的名字是《滔滔千里心》,歌詞充滿「血永遠是濃,永教我激動」等民族主義措辭。事源一九九一年中國華東發生嚴重水災,數以百萬計的災民無家可歸。香港的演藝界很快便動員起來,除了主題曲之外又拍攝籌款電影《豪門夜宴》,由大量香港紅星義務演出,即使每人出場只有十數秒的時間也義不容辭。短短十天時間,香港各界籌款捐助中國大陸總額達四億七千萬港元。

% \begin{figure}[ht!]
%     \centering
%     \includemedia[
%     width=0.6\linewidth,height=0.45\linewidth,
%     activate=pageopen,
%     flashvars={
%         modestbranding=1  % no YT logo in control bar
%         &autohide=1       % controlbar autohide
%         &showinfo=0       % no title and other info before start
%     }
%     ]{}{https://youtu.be/IpWZzKTis0k}   % Flash file
%     \caption{《滔滔千里心》1991 群星合唱 周禮茂填詞}
% \end{figure}

兩年後,香港演藝界更在北京人民大會堂舉辦「減災扶貧創明天」義演,為中國大陸偏遠地區的發展籌款。相對於八九民運,這些行動同樣能夠讓當時的香港人表達與中國的聯繫,政治上卻安全得多,甚至可以建立官方的友好關係。另一個比較,是一九九六年的民間保衛釣魚台運動,當是中港台聯手民間合力突破日本海上保安廳的阻撓登上釣魚台,同時舉起五星紅旗、青天白日滿地紅旗,以及香港市民簽名的橫幅。這次事件背後雖然同樣顯示了九七前香港人對中國人身分的尋覓,也同樣有人動員香港演藝界支持,但因為政治上較為敏感,演藝界當時的反應就明顯地不熱衷。

演藝界的轉變代表了整個香港社會在九十年代面對的一個巨大轉變。鄧小平於一九九二年南巡,重申支持改革開放。及後中國大陸引進大量海外直接投資,每年總額在一九九二至二千年期間的平均額為三百五十三億美元,而在一九八九至一九九一年間此數僅為三十七億美元。當時引入的大多數投資都是來自香港的資金,目標在珠三角一帶。以廣深高速公路為例,就由香港合和公司投資建設,至今被稱為「中國最賺錢的高速公路」。港商更普遍的投資,則在於把本來設於香港的輕工業生產遷到珠三角,例如成衣和塑膠等勞工密集的生產,以求利用中國大陸廉價的勞動力來降低成本。香港本身的主體經濟則變成為這些生產提供金融、銷售和物流服務,也就有所謂「前店後廠」的說法。

去工業化的過程為香港社會底層帶來不少震盪,藍領工人就要面對轉業困難的問題,但對於社會上層來說則是迅間暴富的大好機會。在這個大環境之下,中國大陸不再是讓香港人感到焦慮不安的壓迫者,而是一個龐大而有待開發的掏金天堂。香港人擁有的資金、管理經驗和海外聯繫,正正是改革開放初期中國大陸所最需要的。畢竟,這時候歐美國家的資金尚未大規模進入中國大陸,香港資本扮演起先行者的角色。在此,香港一直以來「從邊陲改變中國」的地位得到另一次轉世再生,愛國、愛港和愛資本成為共識;既然賺錢是香港人的強項,到中國大陸投資就是香港人一種理直氣壯地成為中國人的方式。在這種「資本就是愛國」的熱潮下,瘋狂得就連香港男士到東莞尋求性服務也會自認是愛國行為,還會戲稱自己是「救國跑兵團」。

當時香港的社會和文化研究學者把這個現象總結為「北進殖民主義」。面對主權移交,香港人不單止不用擔心身分認同被弱化,反而能變得前所未見地強大,因為香港身分認同和中國身分認同不再矛盾,整個中國都會成為香港人的舞台。中國大陸成為解決香港各種問題的出路,例如面對香港的樓價問題就有發展商在珠三角廣建別墅,然後以「有身份證就可以做業主」的口號在香港促銷。相對於上文自卑自憐的「夾縫論」,「北進殖民主義」的出現沒有構成衝突,反而兩者更是互相呼應,為香港人的認同糾結提供精神寄託。畢竟,自卑和自大從來都是一體兩面的,無論在甚麼地方都一樣。

對於這種「另類愛國」,當時的中國政府無任歡迎。中國政府要面對的,除了是要引進外資持續改革開放之外,也要處理主權移交後的香港管治問題。之前英國對香港實行間接管治,在不開放管治權的前提下,主要靠籠絡本地精英以維持社會穩定,以求一方面在表面上容許本地人參與,同時又不用失去實質的決定權。中國政府理解英國在港的這套管治模式,並在一九九七之後將之據為己用,而港商在九十年代於中國大陸的投資熱潮,正正為中國政府收編這些本地菁英提供大好機會。

回到中英就香港前途談判期間,不少本地菁英對前途十分擔憂,不單舉家移民領取外國護照,企業也預備遷冊分散投資。當中國政府眼見他們到了九十年代有意到中國大陸「掏金」,便樂意提供各種投資機會和優惠,以換取他們在政治上的忠誠,讓他們成為在香港的政治代理人。如是者,這些本地菁英紛紛改變政治認同,政治圈中也有所謂「老愛國不如新愛國,新愛國不如忽然愛國」之說。而正如港英政府懂得在香港設立各種委任制度以所謂「行政吸納政治」的模式把本地精英納入建制之內(見問題二十),中國政府也照辦煮碗成立了各種職位,如「預委會」、「籌委會」、「港事顧問」和「區事顧問」等等,收納香港社會各界代表。

要看香港社會的極速轉向,可參考九十年代中期的兩齣廣告片。當時流動電話在香港剛開始普及,不同流動電訊商都會大灑金錢在電視台的黃金時間賣廣告。這些廣告按現在的說法可稱為微電影,一般用三分鐘或以上交待一個完整故事,邀請巨星在往往是海外的實景拍攝,而流動電話則以置入行銷的方式呈現。其中最經典的作品,要數和記電訊在一九九四年推出的廣告《天地情緣》。片中主角黎明原是一個極權國家中的情報人員,並結識了獨裁總統身邊的私人秘書。後來黎明發現總統某些他不能接受的秘密,決定現身揭發再投身革命黨,最終被秘密警察抓到。流動電話在廣告中的意義,在於幫助主角和革命黨人聯繫。廣告播出時距離八九民運和東歐變天只有五年,把推翻極權視為英雄故事在廣告商眼中並不是一個問題,更能引起觀眾共鳴。

% \begin{figure}[ht!]
%     \centering
%     \includemedia[
%     width=0.6\linewidth,height=0.45\linewidth,
%     activate=pageopen,
%     flashvars={
%         modestbranding=1  % no YT logo in control bar
%         &autohide=1       % controlbar autohide
%         &showinfo=0       % no title and other info before start
%     }
%     ]{}{https://youtu.be/3rYq7CEzl6A}   % Flash file
%     \caption{《天地情緣》1994 和記傳訊}
% \end{figure}

到了一九九六年,另一個流動電訊商數碼通推出了一齣節然不同的廣告。片中主角周潤發是一名遊走中港之間的政商界人士,在中國各地都會受到熱烈歡迎,當地領導無不感激他帶來的投資和管理經驗。由於他工務繁忙,冷落了家中的妻子和女兒,流動電話成為了他尋回家人關懷的方式。只不過短短兩年時間,廣告中的英雄人物從挑戰極權的革命黨人變成中港推崇的政商名人。嚴格來說,兩者都是在介入國家管治,只是一個推崇自由,另一個利用資本,兩者之別很大程度上就代表了九十年代的中港關係的轉變。值得一提的,是兩齣廣告中都出現了北京人民大會堂,在第一齣廣告中是獨裁總統發表演說的地方,第二齣廣告中則輪到香港人周潤發在台上向全國人民宣告:「中國和全世界將邁向更繁榮的二十一世紀」。

% \begin{figure}[ht!]
%     \centering
%     \includemedia[
%     width=0.6\linewidth,height=0.45\linewidth,
%     activate=pageopen,
%     flashvars={
%         modestbranding=1  % no YT logo in control bar
%         &autohide=1       % controlbar autohide
%         &showinfo=0       % no title and other info before start
%     }
%     ]{}{https://youtu.be/o733CiFz-kk}   % Flash file
%     \caption{數碼通廣告 1996}
% \end{figure}

香港人的自我膨脹,當然也在其他流行文化中反映出來。其中最有代表性的,應是陳百祥的《我至叻》一曲。陳百祥是著名香港藝人,以「阿叻」自稱,而此曲則輕而易舉地把他喜歡自吹自擂的形象拉闊到整個香港在一九九七前的狀態。此曲明顯以「唱好一九九七」為題材,並且大書特書各種香港人主觀自認的優勝之處,如「靈活易適應/勤力手快又眼明/識睇時勢兼淡定/買屋買車仲有錢淨」。而對於香港處於中國和世界之間的位置,則比《這是我家》更毫無保留,赤裸裸地放大為「WEAR銀WEAR過境/閒閒地都講中英/起樓起橋快夾精/鬼佬睇到都眼擎擎」。至於對未來的想像,則既然是「上面咁多金掘/怕你手軟唔去𢴇」,當然毫無疑問高呼「唔駛問阿盲公炳/福星高照好前程」。後來有傳媒報道指出,當年創作此曲原來是受建制陣營委託,目的正正是要「唱好香港」。

% \begin{figure}[ht!]
%     \centering
%     \includemedia[
%     width=0.6\linewidth,height=0.45\linewidth,
%     activate=pageopen,
%     flashvars={
%         modestbranding=1  % no YT logo in control bar
%         &autohide=1       % controlbar autohide
%         &showinfo=0       % no title and other info before start
%     }
%     ]{}{https://youtu.be/KxjZpfhgj1k}   % Flash file
%     \caption{《我至叻》1994 主唱陳百祥 填詞小丙}
% \end{figure}

如此自誇的說法,今天看來明顯是幼稚和無知的,但當時的香港人卻相當接受,陳百祥憑僅此一曲便可以在紅磡體育館開演唱會。不少學者更指出這些九七前香港人有選擇性的歷史回憶和未來展望,本身隱含許多價值判斷,甚至阻礙社會自我反省。學者馬傑偉在分析九七前以總結香港歷史為題的匯豐銀行廣告時,就指出廣告中隱含強烈的個人主義,歌訟「香港地、靠自己」,雖然很切合香港人自以為香港的市場經濟相對於中國大陸的計劃經濟優勝的想法,卻同時把香港的所謂市場制度過於美化,把本身各種制度上的不公平隱藏起來。

九七前香港主流社會中那些毫無羞愧之心的自我陶醉並沒有被任何力量禁止,社會各界到了一九九七年都沉浸在「明天會更好」的樂觀氣氛當中,對未來的懷疑很容易被各種或真或假的利好消息所蓋過。對於英國政府來說,為了顯示他們把香港以最佳的狀態交到中方手中,做到了「光榮撤退」,自然不介意能把香港說得有多好就多好。對於中國政府來說,「香港回歸」是中華民族復興的頭等大事,代表中共領導下的國家強大,有利建構執政認授,同樣也不介意能把香港說得有多好就多好。至於香港人自己,也盲目相信中共基於面子問題,不可能讓香港變得比以前差。就在這全面利好的氣氛之下,香港來到一九九七年七月一日 ,香港特別行政區成立的一天。


伸延閱讀:
「北進想像」專題小組(1997):〈北進想像:香港後殖民論述再定位〉,陳清僑編《文化想像與意識形態:當代香港文化政治論評》,香港:牛津大學出版社。

馬傑偉(1999):《香港記憶》,香港:次文化堂。