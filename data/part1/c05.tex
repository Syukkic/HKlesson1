\hyperref[sec:sec13]{\section{香港認同和中國認同是否對立?}}
\label{sec:sec5}

香港認同和中國認同本來不一定互相排斥。在香港歷史的絕大多數時間中,香港認同和中國認同有著微妙的共生關係。香港認同既和中國認同相關,又刻意有所區隔。這種「既中國又非中國」的態度,在不同時代和環境各有演譯。

首先,所謂中國認同在香港人的眼中可以指涉甚為不同的理解:它可以是指古典中國的文化歷史傳統,也可以是指政治上對共產黨或國民黨的效忠,亦可以是指社會主義中國以至改革開放後中國的生活經驗,三者既可相依亦可相距,按不同時代和人群而異。例如不少香港人會一方面以繼承中華傳統文化為榮,同時認為無論共產黨或國民黨都不能代表中國。事實上,早在民國初年,港英政府意識到中國的政治混亂可能會影響到香港的穩定,於是刻意在香港的教育制度當中多講中國傳統思想文化,以在香港社會建立一套超越當代中國政治的中國認同,而這點在二次大戰後國共鬥爭熾熱的歲月變得更為重要。

至於對香港本身的認同,上文提及到了七十年代末,香港已和原來的難民社會有明顯分別。難民社會的特徵是有一大批人被迫遷移,而到達後因為人地生疏,往往經歷社會地位向下流。為了討生活,他們大多不介意重新開始,把向上流動的渴望放在下一代身上。這不是說他們就沒有不滿,也不懂得抗爭。事實上香港歷史上最暴力的抗爭正是在之前的難民時代發生的,如前文提到的「雙十暴動」。然而對於大多數人來說,他們談不上對香港有太多的歸屬感,只為仍然存活而興幸。不過當這個社群穩定下來後,新家園的建立就趨生了新的認同。

這種「一起重新開始」的認同感,在七十年代末到八十年代初的流行文化中特為普遍。以一九八一年無綫電視劇集《前路》為例,就以主角從中國大陸逃到香港,在香港爭扎求存的故事為題。該電視劇的主題曲《東方之珠》由甄妮主唱,當中副歌一段「若以此小島終身作避世鄉/群力願群策/東方之珠更亮更光」可謂時代寫照。

% \begin{figure}[ht!]
%     \centering
%     \includemedia[
%     width=0.6\linewidth,height=0.45\linewidth,
%     activate=pageopen,
%     flashvars={
%         modestbranding=1  % no YT logo in control bar
%         &autohide=1       % controlbar autohide
%         &showinfo=0       % no title and other info before start
%     }
%     ]{}{https://youtu.be/JshDn96b0vI}   % Flash file
%     \caption{《東方之珠》1981 甄妮主唱 鄭國江填詞}
% \end{figure}

同期類似的作品還有一九七九年的無綫電視劇集《網中人》和《抉擇》,故事大綱都是新移民在港重新起步的故事。當中《抉擇》由林子祥主唱的主題曲,可謂描述移民落地生根心態轉變的經典。

\begin{displayquote}
    幾多往時夢 幾許心惆悵
    別了昔日家 萬里而去 心潮千百丈

    收起往時夢 拋開心惆悵
    任那海和山 助我尋遍 天涯各處鄉

    闖一番新世界 挺身發奮圖強
    要將我根和苗 再種新土壤
    
    就算受挫折也當平常 發揮抉擇力量
    再起我新門牆 似那家鄉樣(勝我舊家鄉)
\end{displayquote}

% \begin{figure}[ht!]
%     \centering
%     \includemedia[
%     width=0.6\linewidth,height=0.45\linewidth,
%     activate=pageopen,
%     flashvars={
%         modestbranding=1  % no YT logo in control bar
%         &autohide=1       % controlbar autohide
%         &showinfo=0       % no title and other info before start
%     }
%     ]{}{https://youtu.be/FwP9jNpQ2rg}   % Flash file
%     \caption{《抉擇》1979 林子祥主唱 黃霑填詞}
% \end{figure}

這首歌的填詞人是香港流行文化的一代鬼才黃霑。按他生前所述,這首歌是他逾千首作品之中最喜歡的一首,甚至比後來被譽為香港非正式代表歌曲的《獅子山下》更甚。黃霑原名黃湛森,一九四九年時僅八歲的他隨父親從廣州逃到香港。對他來說,前來香港的這個「抉擇」改變了他的一生。而他的成就和貢獻,都是「發揮抉擇力量」所帶來的。這首歌只有短短一百字,已說明他身為移民對香港的感情和盼望。其中副歌最後一句,第一次唱的時候是「似那家鄉樣」,對故鄉仍有思念之情。但到了歌曲完結時,這句卻變成「勝我舊家鄉」,說明新的認同感已通過在香港過新的生活建立。

值得注意的,是這些以「從新起步」為題的文化產品當中,往往會用到「天涯」和「小島」等的詞語來描述香港,並把這些詞語演譯為正面描述。這種取態,可稱之為一個從難民心態走向「舢舨想像」的改變。舢舨是華南常見的平底木船,通常沿岸航行,經不起大風浪。「舢舨想像」是指那些一同逃到香港的難民及其後代意識到他們分享同一種的經歷:香港是他們離開中國政治風暴的一隻救生艇,他們能夠在香港這個比當時的中國大陸相對安全的地方爭扎求存,要好好互相幫助,以求同舟共濟。如在《獅子山下》一曲當中,就有「同舟人誓相隨/無畏更無懼」和「同處海角天邊/攜手踏平崎嶇」兩句打動人心的名句。在此,香港在中國的邊緣位置成為一件香港人賴以為生甚至是可引以為榮的事情。在臨近一九九七的時候,香港話劇團、香港中樂團及香港舞蹈團聯手創作一齣以回顧香港歷史為題的音樂劇《城寨風情》,當中的主題曲也有一句「山窮將山擴/獅山有金光/水盡碧海之濱建天堂」,同樣讚訟香港的邊緣位置為香港人留下一條生路。

% \begin{figure}[ht!]
%     \centering
%     \includemedia[
%     width=0.6\linewidth,height=0.45\linewidth,
%     activate=pageopen,
%     flashvars={
%         modestbranding=1  % no YT logo in control bar
%         &autohide=1       % controlbar autohide
%         &showinfo=0       % no title and other info before start
%     }
%     ]{}{https://www.youtube.com/watch?v=edrmgTqw5G0&feature=youtu.be}   % Flash file
%     \caption{《獅子山下》1979 羅文主唱 黃霑填詞}
% \end{figure}

這種邊陲與中原相對應的比較,明顯反映了香港人身分認同的誕生,和香港人如何看待中國大陸不能分割。正當香港在七十年代快速現代化的同時,中國大陸卻陷入文革混亂,更經常有武鬥的死者被「五花大綁」掉落珠江,屍體隨潮水沖到香港沿岸,向香港人提醒中國大陸的混亂和香港的相對穩定之別。到了文革結束,中國大陸的民眾終於有機會接觸外面的世界,香港和中國大陸的社會差距已經拉得很遠。前文提到的電視劇《網中人》,當中由廖偉雄飾演的中國大陸新移民「程燦」,其土氣形象便象徵了香港和中國大陸的差別,「阿燦」也成為不少香港人對中國大陸移民帶有貶義的稱謂。

流行文化對建立香港身分認同可謂功不可沒。六十年代啟播的無綫電視成為重要的大眾誤樂,上述的電視劇成為香港人日常生活的一部份。當大多數人都看同一套的電視劇,唱同一首的粵語流行曲,一種自成一格的認同感便由此而來。學者馬傑偉認為六、七十年代的流行文化「吸收了西方的成份,轉入中國文化的特性,訴說了本地的經歷,凝聚成一種獨特的港式生活」。

說回香港流行文化中的中港差異,類似的描述可謂歷久不衰,例如在一九八九年初上映的電影《合家歡》,主角許冠文飾演一名香港基層家庭的大陸親戚,因為不諳港人文化而處處闖禍,例如以為男廁尿兜的自動沖水是供人洗手之用。不過,儘管這個時期的各種流行文化產品時常以嘲笑大陸移民為題,卻同時不否認香港與中國大陸的感情連結,通常最後都以大團圓結局告終。在《合家歡》當中,許冠文本來被香港家人利用騙取保險賠償,最終各人良心發現明白親情更為重要。

刻意拔高香港同時貶低中國大陸的二元對立,可說是香港認同一直以來的重要元素。來到八十年代初,中英就香港前途問題展開談判,香港社會人心惶惶。正當香港認同要建立起來的時候,旋即就要面對重回中國管治,加上當時的中國只是剛剛走出文革陰影,普羅大眾難免對前途失去信心。如何排解這個信心危機,和解決香港認同與中國管治之間的衝突,就成為當時社會的一大課題。直接一點說,當香港人無從選擇自己的未來,便只好改變自己的心態,一方面嘗試保持和突顯香港的獨特性,同時想像香港和中國的恰當關係。

前面提到的填詞人黃霑也看到這個問題,而他的答案就是《這是我家》一曲。一九八六年,英女皇伊利沙伯二世訪問香港。當時距離《中英聯合聲明》的簽署只有一年多的時間,其訪港有明顯的象徵意義,要表達英國政府對處於過渡期的香港仍然會承擔責任。到了訪港行程的最後一天,香港政府在紅磡體育館舉辦了香港青年精英大滙演,壓軸演出的歌曲就是黃霑填詞的《這是我家》。

% \begin{figure}[ht!]
%     \centering
%     \includemedia[
%     width=0.6\linewidth,height=0.45\linewidth,
%     activate=pageopen,
%     flashvars={
%         modestbranding=1  % no YT logo in control bar
%         &autohide=1       % controlbar autohide
%         &showinfo=0       % no title and other info before start
%     }
%     ]{}{https://www.youtube.com/watch?v=wu3AXf_J2Mg&feature=youtu.be}   % Flash file
%     \caption{《這是我家》1986 群星合唱 黃霑填詞}
% \end{figure}

儘管這首歌是在英女皇訪港的大滙演中演出,但畢竟英女皇不懂得廣州話,這首歌的獻唱對象明顯不是她本人。這首歌的對象是電視面前數以百萬計的香港市民,而歌詞的內容正正是要為香港認同作出定義。首先,歌曲是以中國作曲家王洛賓的民曲(如青春舞曲和康定情歌)為基礎重新編曲,再以廣州話譜上新詞,本身就反映了香港社會的華人文化基礎。歌詞方面,大多以香港當時的快速發展為主題,如「地鐵飛奔到觀塘」和「東區快車湧去走廊」,也回應了香港當時連同南韓、台灣和新加坡合稱「亞洲四小龍」的地位。但是黃霑並沒有停留在物質生活的描述,而將之提升到精神層次,歌訟香港人勤奮向上的精神。在此,他利用了前文提到主流印象中的中港二元對立,用歌詞形造一個自由和充滿活力的香港,暗中與當時凡事講政治立場,讓人感到刻板落後的中國大陸相對應。他甚至毫不忌諱地在歌詞中放入「維園自由唱」的說法,把維園作為香港言論和表達自由地標的角色點出。

不過此曲最激進的地方,還要說到副歌一段:

\begin{displayquote}
    這是我家 是我的鄉
    是民族世界岸
    是我的心 是我的窗
    是東方的新路向

    歡歡喜過日 開開心渴望
    中西客 香港客 攜手合唱
\end{displayquote}

短短四十九字,黃霑重新定義了香港的世界定位。首先,香港不再只是香港人的家,更是香港人的鄉,這和過去難民社會中對中國故有的鄉土追思有明顯差異。與此同時,香港卻又是「民族世界岸」和「東方的新路向」,當中所指的固然是中華民族。在英女皇的面前一大群的香港人高唱自己能夠如何貢獻中華民族,本來應該是一件很奇怪的事情。但在這特定的時空背景,卻明顯有助穩定民心。此時的香港人既認為香港和中國大陸有別,卻同時不否認兩者之間的情感紐帶。如是者,一方面把中國大陸定義為封閉落後,另一方面把香港定義為打破這個封閉落後的窗口,香港人自持的先進性就不用害怕因為連接中國大陸而失去,反而成為優勢,進一步強化香港認同。

把香港自視為引導中國大陸走向世界的領航者,在今天的目光來看未免狂妄自大。然而回到八十年代的中國大陸,這想法卻不是那麼的不切實際。首先,當時的中國大陸無論從經濟和社會發展方面確實都遠比香港落後,中國大陸的改革開放極為需要來自香港的資金和信息網絡,香港的確扮演了中國和世界的橋樑的角色。與此同時,香港的流行文化如電影、電視劇和流行歌曲,在中國大陸還未建立起自己的偶像系統之前也是大賣特賣。如果把文化二字的定義推廣到管理制度,當時的中國大陸更興起標榜「港式管理」,與過往國有企業不重視以客為尊的服務態度區分開來。

更重要的,是當時的中國大陸確實希望走向世界。八十年代中期的中國大陸,可謂近代中國最為開放的時代,各種思潮從世界各地湧入,而中國大陸對這些外來的想法都抱有開放的態度。當時流行說「與國際接軌」,這句話本身就意味承認了國際標準的存在,而且是中國大陸要學習的對象。這和今天的中國大陸強調的「三個自信」,即「道路自信、理論自信、制度自信」,有差天共地的分別。中央電視台在一九八八年播出的紀錄片《河殤》,更是大力批判中華傳統文化固步自封,認為中國的未來在於擁抱「藍色海洋」文明,引發當時社會轟動熱議。當時中國大陸的言論空間之廣闊,社會對不同未來的想像和探求,能予人感覺充滿希望。

在這一系列的時代背景之下,黃霑把香港說成是「民族世界岸」就不是那麼純粹的空想和自我安慰。既然主權移交是無可避免的,那麼香港人的最佳選擇除了離開香港,就是用盡各種手段把中國大陸變得更像香港。這樣下來,說不定到了一九九七年的時候,無論是文化、經濟或社會習慣各方面,都會是香港收回中國而不是中國收回香港。

總的來說,回到七、八十年代香港認同剛成形的時候,香港人並沒有全盤否定中國認同,而是更大程度上是以「不一樣的中國人」自居。然而當相對開放的政治環境消失,香港認同以及其與中國認同的關係也無可避免改變。


伸延閱讀:

馬傑偉(1996):《電視與文化認同》,香港:突破出版社。

張美君(1997):〈回歸之旅:八十年代以來香港流行曲中的家國情〉,陳清僑編《情感的實踐:香港流行歌詞研究》,香港:牛津大學出版社,頁 45-74。