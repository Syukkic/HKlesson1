\section*{前言:答問作為一種異議}
這個寫作計劃的緣起,要說到八年前獲香港中文大學新聞與傳播學院邀請,我在他們的碩士班開課講香港社會與政治。學院的碩士班和香港其他修讀式的研究院課程一樣,來自中國大陸的學生很多。他們一方面未必有社會科學的背景,對香港的議題也一般不甚了解,於是學院便想到開一門公共事務分析課,並以香港為案例,培養他們當記者所需的社會思考意識。我立即同意了學院的邀請。與從中國大陸來香港諗新聞的學生談香港,可是一件好差使。

儘管願意來香港諗新聞的學生通常比較崇尚自由,但要和他們談香港也不容易。畢竟,中國大陸對香港有其官方敍事,對於從小在這套系統內認識香港的學生,來港後往往會遇上很多現實差距,每件事情都要逐一解拆。例如近年來香港人的中國認同屢創新低,自認中國人的年輕人曾跌到百分之一都沒有,這點來自中國大陸的學生大多沒有聽過,也很難立時理解。

中港如何認識彼此和自身,這其中的差異其來有自,近年更變本加厲。在當前中國的官方論述當中,強調中國在晚清之後經歷的「百年屈辱」,直至改革開放之後走進最好的時代。這說法當然有一定基礎,但當社會中的各種事物都被放進這個框架來詮譯時,就會出現很多問題:過去中國曾經更為開放進步的面向,會被抹殺;當今中國尚未解決的問題,會被淡化。

香港在這個宏大論述中被給予了角色:英殖香港被視為中國過去不光彩的一頁,香港回歸就是中華民族復興的證明;當今香港尚會有人批評中國就是執迷不悟,而且一定是受境外敵對勢力唆使。現在的香港,引用《環球時報》的說法,是年輕人被「放縱太久」和被「外國反中亂港勢力」煽動,我們應該認識和批判他們的「險惡用心」和「丑陋嘴臉」。

這些當然都是嚴重片面甚至失實的說法,事實比這些愛國措詞要複雜太多。但說這些話的人要的不是現實。借用近年的說法,中國大陸官方媒體所描述的香港,基本上都是「假新聞」,目的不是要真誠解釋香港在發生什麼事情,而只不過是為既有的政治立場鳴鼓開路。最起碼,當解釋的對象不是香港,或者時空背景不一樣時,同一套路就會忽然消失。愛國在此只不過是政權的工具。

面對情感澎湃的國仇家恨,我總會在開學的第一課和學生說:當你進了這教室的門,你的第一身分是一名學者,一個獨立思考的人,不是香港人或中國人或什麼人。我歡迎你分享你的個人經歷,我們可從中互相學習;但一個觀點是否值得支持,我們只應看它在學究上是否站得住腳,更千萬不要因為結論不合脾胃便立即聯想到一定是對方動機不純。回溯真實,判斷是非,與國族身分無關,和愛國與否無關,也和結論傾向哪一方無關。學術就是學術。

問身分認同,近年文化研究的學術共識是,儘管它或有一定的物質基礎(例如隔個山的社群說話口音通常有些分別),但選擇把哪一些文化連結放大,把哪一些的文化分野抹去,以便製造共同的向心力,就是社會使然的政治工程。這樣去理解,比談「血濃於水」或「民族大義」更能解釋自二次大戰以來香港人身分認同和中國想像的離合。

問政府管治,近年政治學認為要明白一個地方的管治得失,除了要看組織制度與法規條文外,也要研究它們是在怎樣的時代背景之下訂立,又在怎樣的時代背景之下執行,而兩者之間的差異又會如何使得同一套的組織制度與法規條文可帶來不一樣的社會影響。以此理解什麼是一國兩制和高度自治,比重複「香港自古以來就是中國的領土」更有助釐清香港政治爭拗的重點。

問經濟發展,近年社會學強調市場並不抽象地存在,而是嵌入在特定的社會環境當中,優勢和弱點不一定來自先天或個人因素,也很多時候受所處的權力關係所影響,有絕對權力者甚至可以製造壟斷,所以我們不能假設有買有賣就代表市場有效運轉。以此來理解香港的房屋和勞工,甚至醫療與教育等議題,比迷信「獅子山下」以為努力總有出頭天來得實際。

上述總總,在學術界談不上是什麼大道理,就算是大學一年班的學生也應該聽過。這些分析方法,也不止可應用於香港。如果學生畢業後有日被派到南美洲去報道新聞,用上同一系列的社會思考意識,也應能更有深度。

於我而言,並不關心學生在這課程完結後對香港的政治立場有沒有改變,重點是他們知道認識社會議題需要的方法和態度,可以獨立思考,不要人云亦云。

從求學問出發,無論你如何看不起「左膠」、「笨土」或「小粉紅」(香港的左翼社運參與者、本土主義者和中國大陸民族主義者的網上標籤),也得先客觀地理解他們存在的社會基礎,是什麼促使他們出現。誠然,香港有些人抵制中國大陸的言行明顯地不理性,反之亦然,但這不代表我們就可以漠視這些現象背後的原因。如果我們因為情感上不喜歡某些行為,便將其動機也同時否定,那麼我們自身也不見得很理性。

強調立足學究,是因為我不希望我在這裏提供的回答會被用來助燃更多的中港矛盾。見過太多所謂的討論倒退成「你不是香港人,我比你清楚」或「你不在中國大陸長大,你不會明白」之類唯出身論的說法,太沒意思。從我的教學經驗來看,在香港長大的學生對香港各種情況其實並不見得遠比中國大陸的學生更掌握。反過來說,中國有十四億人口,即使在中國大陸長大的若要自稱熟知各處情況也未免過於自信,更別說各種信息審查和歪曲的影響。

為免被自以為是所矇騙,社會科學常常強調要學習當一個「專業的陌生人」,以超越日復一日的爭吵。見到現象,不用馬上批判,更重要的是去追尋原因。這樣把熟悉變陌生的做法,對中港都有好處。很多香港人也許已經習慣或覺得理所當然的東西,其實需要更多解析。例如立法會議員終日吵吵鬧鬧,很多香港人自己都感到煩厭。但抽離去看,不難發現吵鬧背後的結構,或不理性背後的理性。反過來,一些在中國大陸已成習慣或普遍覺得理所當然的事情,也可被香港這個特例顛覆。例如「香港人」這個概念從出現到今天短短數十年來已轉了很多很多道彎,身在香港不用熟讀學術經典都知道國家、民族,和政權是三個相關卻不相等的概念。就算有日香港在地圖上消失,各種香港的疑問仍應留下,讓中國以及世界上的所有人持續思辯。

雖然強調學究,但我也理解這個寫作計劃的政治含義。

當今香港,講道理已是政治。當政府可以邀請一大群專家學者做數個月的土地諮詢,卻在諮詢期完結前推出一個各項參數設定也和原諮詢不符的方案,無異於告訴世人他們從不在乎任何認真的思辯;而當政府接連這樣做的時候,民間的回應也越來越變為宣洩不滿多於尋求認真答案。在此環境,放慢腳步,把道理說清,已是異議。擴闊一點想,過去數年隨著中國大陸政治環境改變和言論尺度的收緊,對香港的負面宣傳成為了煽動民族情緒和鞏固政權管治的手段之一。那麼,把香港的不同面貌表達出來,也是異議。

正正基於這政治含義,我預期我在此提供的問答會受到很多挑戰。在學術世界,同一件事情有不同的理解不足為奇,知識就是在爭辯中成長的。如果你拿著我提的問題去找另一位學者,他或她說不定會給你另一系列的解答。我可以做的,是為我的每一篇解答(以及這篇序言)提供伸延閱讀,讓各位知道討論脈絡的源起,也可以按此進一步研討。

至於研討過後,我不尋求必然的共同結論,只望能對講道理有所追求。畢竟對於本來就不打算講道理的,我再寫十萬字也不可能說服。近年有些觀點聲稱為了中國的崛起,面對香港(或台灣)時不用考慮當地人的感受,直接移平推倒重來就是,也就是所謂的「人滾地留」。這種說法基本上和納綷德軍的東歐政策沒有分別,如果有人要持這種立場來討論問題的話,要處理的恐怕已不再是論證而是基本價值觀甚至是人性問題了。

回想二零一四年的佔領運動,我在佔領爆發當天寫了一篇後來被稱為「佔中十七問」的答客問,解釋這場運動爆發的政治背景。有市民將該文大量印刷拿到佔領區派發,也有網民轉貼在網上廣泛流傳,協助其他地方的人理解佔領運動的前因。出版過後,不少台灣媒體特別來找我做專訪,文章也在中國大陸的網絡審查下流傳,還引來官方以「香港佔中十問」回應。

佔領運動雖已遠去,香港故事還未說完,對香港的扭曲誤讀還在繼續。接下來的十萬字,可理解為當天辯解的一個延續。辯解,不僅僅是為了香港本身。以扭曲事實來服務政治,以情緒指控來愚弄大眾,在中國乃至世界都不罕見。按其他題材以答問來提出異議的人,中國還有很多很多,他們的處境遠遠比我困難。接下來的答問,也算是向他們致敬。

伸延閱讀:

高馬可(2013):《香港簡史—從殖民地至特別行政區》,香港:中華書局。

徐承恩(2016):《城邦舊事:十二本書看香港本土史》,香港:紅出版青森文化。