\hyperref[sec:sec30]{\section{中港對特區政治制度的最大分歧是什麼?}}
\label{sec:sec14}

中國政府和香港社會對於特區政府的認授來源有根本分歧,香港各種政制爭議歸根究底均由此而來。特區政府的認授在中國政府眼中近乎不證自明,但對很多香港人來說卻是要建立和維持。此分歧使得中港各種政治問題上糾纏不休,其中以普選和國家安全議題特別明顯。

所謂認授,簡而言之,就是要回答一條簡單的問題:「你憑什麼可以管治我」。政治學所關心的議題,很多時候都涉及人類社會如何建立、維持和修改各種規則。畢竟我們並非居於荒島,只要有多於一個人就要有規則,然後就要問誰有權決定這些規則,其他人為何要服從,而當有人拒絕服從時又用什麼方式實施或重訂規則。

最赤裸裸的做法是以武力解決,但這樣做很沒有效率。試想想,如果政權每次要求民眾或公職人員做一件事情,都要找個有槍的人出來走一圈去耀武揚威,這個政權得養多少個有槍的人才能解決問題?再者,這些有槍的人也有可能搞武裝政變,他們的監管也是一個問題。與此同時,那些民眾和公職人員就算服從也很可能會陽奉陰違,不會真的把事情做好。比較有效率的做法,是要讓他們都心悅誠服地追隨和服從政權。換一個中國的傳統說法,就是孔子在《論語.顏渊》提到的「民無信不立」。

要建立認授,傳統來說可以靠神話,例如皇帝可以說自己是天子,也就是君權神授。但萬一有另一個人走出來說他才是天子,而當時又發生一些可理解為上天改變主意的徵兆,例如地震或者洪水,那麼認授就會受到挑戰。另一種做法是靠個人魅力,例如有些人是演說能手,很善於說服其他人去追隨他,同樣可以建立認授。然而人的生命是有限的,到了這個領袖百年歸老之後,他的接任人如果沒有同等的魅力,政權以外卻出現另一個很有魅力的領袖,則同樣會產生不穩。我們也可以拉闊一點,把個人魅力轉化成意識形態的魅力,說服民眾一起為某個理想去奮鬥,所以要跟隨某一群人。但這種認授要得以維持,就要不斷提供現實或虛構的例證去說明社會正在一步步的走近那個理想。如是者,認授最終還是關繫到政權的表現。

現代政治學明白到沒有一個政權可以永遠維持認授,畢竟天災人禍總會發生,社會發展也不會只向前而從不衰退。所以問題的核心,是當政權出現認授危機時,如何把權力以非戰爭、內戰或流血政變等的方式和平轉移,減少對社會的衝擊。這就是民主制度的根本。當權力和平轉移了,新的政權就得到新的認授。而被拉下台的人,也可以努力通過選舉重新奪得政權。當大多數人都認同這個制度,相信自己有份決定政權誰屬,認授更新的過程就不用那麼你死我活,社會就變得穩定,不用刻意維穩,減輕管治成本。

再簡單一點說:沒有認授,人民會造反。講認授,就是認清不是所有的規則都應當遵守,也不是所有的權威都應當尊從。如果有人衝擊圍牆,我們應先問這圍牆為何存在,而不是譴責一切衝擊。

把這個討論放在香港,嚴格來說九七前的港英政府同樣要面對認授挑戰。他們既為外來政權,和本地社會交往不深,也不容許本地人取得管治實權,理應受到質疑。有三個理由讓港英政府受到的壓力得以減緩。第一,當時香港接收大量來自中國大陸的難民。他們自願選擇來到英國管治下的香港,在他們眼中中國大陸的形勢比香港更糟糕,對港英政府的要求也因而降低。第二,香港經濟在戰後快速發展,而且比中國大陸穩定和開放,港英政府可通過表現來維持認授。第三,到了後過渡期,雖然隨著本地公民社會走向成熟,社會對參與政治的要求越來越多,但對九七的期望或恐懼卻更為重要。既然中國政府已答應了九七後是港人治港,於是對港英政府的認授挑戰,很容易便會轉化為對未來特區民主化的追求。直白一點說,即使有不滿也好,反正英國人數年內就要走了,對英治的挑戰也同時消除了。

但特區政府就不一樣了。如前文所述,香港人在主權移交前夕基於各種原因處於不現實的亢奮狀態(見\hyperref[sec:sec7]{問題七}),所以對特區政府本身也有極高期望。當對特區政府的表現不符期望,市民開始時只對個別的領導人物(如董建華)表達不滿。但隨著年月和人物替更,特區政府的表現仍然未有改善,而港人治港的普選承諾又未能實現,公眾便會開始質疑制度本身。換言之,當香港人眼前的問題越來越多,又沒有未來的日子可期望,認授危機便隨之爆發,也就回到一開始的那條問題:「你憑什麼可以管治我」。有時中國大陸的輿論會質疑香港的政治問題只不過是政客操弄民意所致,但想深一層,他們有誘因並能夠這樣做,本身也說明社會出了問題。正確的處理方向,應該是面對社會本身的問題,正本清源。

特區成立以來的各種管治危機,社會大眾無法團結一致,後面都是認授問題。而特區政府的認授問題,核心在於特區從何而來。

站在中國政府的立場,香港從來都是中國的一部分,中華人民共和國在一九九七年七月一日恢復行駛對香港的主權,並按中國《憲法》第三十一條設立香港特別行政區。換言之,兩制是由一國所創立的,特區的權力和地位在制度上是由全國人民代表大會說了算,給多少就有多少,不存在認授問題。

站在香港社會的立場,儘管很多香港人在情感上和中國有連結,但這個「中國」很多時候是一個文化上的虛銜,在他們眼中和中華人民共和國這個實際政權並不對等。在八十年代初,大多數香港人都拒絕接受中華人民共和國的統治。後來慢慢願意接受,很大程度上建基於一國兩制的保證,即相信九七後能和中華人民共和國的直接統治保持一定區隔。換言之,對於這些香港人來說,沒有兩制,一國是無從談起的。而兩制的創立,是由《中英聯合聲明》所確定,當中詳列中華人民共和國政府對香港的基本方針政策。與此同時,《基本法》又明文規定中國政府在任何情況下也不能偏離這些載於《聯合聲明》的方針,而《聯合聲明》本身又在聯合國備案。在這些理解下,中華人民共和國在香港的地位並非不證自明,而是一種社會契約,最起碼中港雙方都各有責任。反過來說,當中方不再按其承諾行事時,例如做出違反《聯合聲明》和《基本法》的事情時,則最起碼在道義上的層面,香港作為中華人民共和國一部分這點就會開始受輿論質疑了。

簡而言之,對於中國政府來說,是一國的基礎創立了兩制;而對於香港社會來說,卻是兩制的承諾帶來了一國。一邊是一國先於兩制,另一邊是兩制容許一國。中國政府和香港社會一開始就有這兩種差天共地的出發點,香港政制的各種問題都可以說是從這差距而來。由於在中國政府的立場來說特區政府根本不存在認授問題,也就不用解決;香港社會卻認為特區政府表現不濟而且沒有認授,要改變政治制度才能根本地扭轉問題。香港人排拒特區政府,實為認授危機的病徵。而這個先天的缺陷,隨著行政權、立法權和司法權在特區成立以來的逐步弱化而不斷擴大,使得香港陷入有些本地學者視為「不能管治」的境況。

\rule[-10pt]{15cm}{0.05em}

伸延閱讀:

Scott I (2007) Legitimacy, Governance and Public Policy in Post-Handover Hong Kong, \textit{The Asia Pacific Journal of Public Administration 29:1}, p.29-49.

Sing M (2011) The Legitimacy Problem and Democratic Reform in Hong Kong, \textit{Journal of Contemporary China 15}, p.517-532.