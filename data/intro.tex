\section*{簡介}
這本來是我在香港中文大學教書時,寫給內地生的課堂筆記。現時每年有約兩萬名中國大陸學生前來香港,修讀本科學位和研究院等課程。他們來到香港後,都會面對各式各樣的文化衝擊。尤其面對刺熱的中港矛盾,當親身遇到香港民意排山倒海的反中情緒時,就算沒有抗拒情緒,最少也會感到難以理解。

我是梁啟智,過去八年(2011-2019)在香港中文大學教授香港社會與政治,在課堂內外見證香港風雲色變。我的學生大多來自中國大陸和世界各地,上課前通常對香港各種困局背後的原因所知甚少。年復一年在驟變中向背景各異的學生解釋「什麼是香港」,我發現儘管社會論爭不休,然而無論是外來者或是本地人也不見得很清楚自己在吵什麼。

對內地生,既然來到香港,不如把整個香港當作是教室,理解差異和衝突的來由;對外地人,看清香港這個中國體制與普世價值碰撞的最前線,有助思考如何應對中國冒起;對本地人,換個角度當一回認真的陌路人,或可成為一種自省的方式。在問香港應往何處去之前,不如我們齊來先退後一步,嘗試把最基本的東西說清楚。