\chapter{結語:論系統性敗壞}

回到二零一四年,佔領運動爆發前一週,各大學剛開始罷課,有從中國大陸來港的學生說想去政府總部「罷課不罷學」的現場聽演說,但又擔心安全。我想他們在大陸長大,相信都沒參與過群眾運動,在新聞聯播上看到的都是剪裁過的外國騷亂場面,維穩話語下成長難免會對參與有戒心。當時我安慰他們說,香港的群眾運動都很和平,零三年的時候五十萬人上街也沒有打翻一個垃圾桶。

沒想到,和平的共識就在佔領運動中被打破。數天後,因為警方的七十九枚催淚彈,香港史上最大規模的佔領運動爆發。而讓我更意想不到的,是公務員獨立專業的神話在這場佔領運動中完全破滅。「旺角黑夜」期間暴徒帶同利器攻擊和平集會的市民和學生,警察卻沒有把他們繩之於法。其後,警察失去了市民的信任,警民衝突不斷升級,香港「和平理性非暴力」的抗爭傳統也受到動搖。

這些年來,我在大學教香港社會與政治,最大的感觸是事物越來越變得違反常識。很多過去以為不用討論、與香港社會不相關的東西,忽然變得重要。例如討論司法制度的那節課,以前主要介紹居港權釋法案,也就是法庭的問題,放在檢控和警察的時間不多。近年來,先是佔領運動,又有「銅鑼灣書店」事件,然後是「一地兩檢」爭議,還有最新近的《逃犯條例》修訂爭議和警察鎮壓公眾集會時所犯的暴行……關於香港司法需要討論的內容大幅增加,原來安排的課堂時間變得嚴重不足。

又例如談到民主化的時候,過去要講香港選舉的不足,主要集中在票值不均等的問題。至於學術文獻中談民主選舉時常提到的「恆常、秘密,和安全」準則,以前只會輕輕帶過,說是落後國家才會有的問題。現在的香港,這三點都成為現實爭議。

這些改變,當然不止在課堂上感受得到。社會輿論近年常常提及的一個說法是「禮崩樂壞」,也就是說許多過去大多數香港人默認可信的社會制度,正逐一瓦解消失。曾經,香港人以為無論你是支持或反對政府,在暴力面前會得到警察的公平對待。曾經,香港人以為無論你的立場如何荒謬絕倫或離經叛道,只要你找到一百個簽名支持和交得起五萬元的選舉按金,就可以去選議員。這些,都已不再能視為當然。

香港病了。治病,得分開病徵和病因。有些病徵,表面上很煩人,但其實是身體發出的善意提醒,叫你是時候注意一下。強行壓下病徵去而又忽視病因,固然不能治病,而且會帶來更大的傷害。可惜,病徵總比病因容易看見。議會亂象、街頭衝突,還有它們所代表的社會撕裂,成為許多人談論香港時的主要甚至是唯一關注。接下來,就會有人開出各種藥方,聲稱只要把在議會抗爭的人趕出議事廳,把在街頭鬧事的人都關進監獄,社會就會和諧,香港才能得以發展。這些想法,當然是太簡單、太天真。特區成立二十二年了,這些藥方使得香港越病越重,卻還有人繼續推銷,看起來更像是推卸責任。

關於香港各種社會現象背後的歷史和制度因由,前文已說了很多,不贅。這兒我想提出兩點猜想,作為這個寫作計劃的結語。

第一,如果制度問題總是一環扣一環,推而廣之,整個社會的崩壞本身又是否一環扣一環?

回到八十年代,當時的香港和今天的中國大陸一樣,以各式各樣的城市建設自豪,政府宣傳片常見地鐵、海底隧道和東區走廊等當代大型基建工程。但近年來,基建卻成為香港社會之痛。地鐵沙中綫和港珠澳大橋工程不單接連超支,更爆出嚴重工程監管問題,令曾經迷信專業至上的香港社會頓感錯愕。相對於民主普選,專業精神可能更應被稱為香港人最引以為傲的核心價值。當偷工減料和質量做假的醜聞接連發生,恐怕對不少香港人來說比各種政治爭拗更能預示「香港之死」。

這倒退是如何開始的呢?專業操守要穩住,既要外在監管也要專業自律。專業自律要發生,首要用人唯賢。如果各專業的領袖本身能力或名望不夠,又或不能做到公正持平,專業自律就會事倍功半。特區成立以來,社會流動階梯越見忠誠先行,而非唯才是用。從行政長官算起,當每一個位置的任命都建基於對上級的服從多於個人能力,專業失效便遲早會發生。

那為什麼忠誠先行會變得越來越赤裸裸?因為權力制衡的基礎改變了。香港雖然沒有普選,但港英時代的香港處於中英二元權力結構當中,港英政府有壓力維持香港的善治。英國人離開了,中國強大了,二元權力結構消失。如是者,權力的邏輯變成由上而下,無論是政府官員或是社會領袖,擔心的只是更有權力者是否相信他們的忠誠。與此同時,專制政權例必不能接受任何獨立於自己的權力來源,德高望重卻又不聽話的民間領袖自然要被換走。如是者,一場劣幣驅逐良幣的浪潮便會捲開,有能力和講原則的只好離開社會上層,甚至離開香港。

一般市民未必天天關心各種政治爭議,卻同樣會被牽連其中。當政府失去管治能力,市民只好自行解決生活所需。政府做不好長幼照顧,就自己請外傭;高等教育學位不足,就去上補習班或者出國留學。然而因為只有擁有財力或社會資本的才能自己解決問題,社會變得更為不平等、更為撕裂。而當大家都習慣把責任個人化,要求在制度上解決問題的壓力會反而降低。市民對政府失去期望,也就不參與公共事務,反對政治也就難以壯大。選舉的時候不投票,親政府的力量更能壟斷議會,會議規則被改得更為自我閹割;議會變得更無用,於是大家更懶得出來投票。當各種原有撐起香港的社會制度都變成鬧劇,絕望之下重建這些制度就更為困難。惡性循環,變本加厲。

以上,我稱之為香港的系統性敗壞。說是系統性,是因為出問題的地方,不是某個個人、團體、政治派別或勢力所做成。是整個香港一起病了。

第二點猜想關於中國大陸。我記得曾經有一位來自中國大陸的同學在學期末的時候跟我說,上了我的課之後對香港感到很同情。其實同情不同情,從來不應該是重點。我在乎的,是同學有沒有通過研討香港問題,養成良好的答問習慣。畢竟,很多事情今天在香港爆發,可能只是比中國大陸早了幾步。同樣的問題,可能很快就要在中國大陸以困難百倍的方式發生,而我很擔心中國社會尚未準備好回答這些問題,甚至未能提供探索的空間。

事情往往以驚人相似的方式發生。中國大陸現在的急速發展,讓很多人過上比之前更好的生活,也為很多人帶來了當中國人的榮耀感。同樣的過程,香港比中國大陸更早發生,也為香港人帶來很多榮耀感。香港的經驗是:高速發展可以舒緩很多問題,但也會隱藏很多問題。到了高速發展結束的時候,這些問題就會一一浮現。

近年中國大陸的經濟成就,讓很多人變得有自信。自信很重要,但過多的自信會蒙蔽自我,這也是香港的經驗。今天的中國,浮游著一種唯經濟、唯發展、唯技術、唯實力的決定論,以為它們可解決一切問題。但香港的經歷說明,這只是特定時空下的產物,終有一天會到達瓶頸。當社會流動開始減慢,當各種問題已不能再通過發展來解決,就不得不面對一些更深層次的問題:價值觀的問題、制度的問題,甚至是上面提出的系統性敗壞。

讓我們暫時放下繁華外衣,平心靜氣的問一問:今天的中國社會,有沒有專業失德的問題?帶來了怎樣的社會危害?原因和做不到唯才是用有沒有關係?後面和中國當前的政治和社會結構有沒有關係?當民眾發現政府不能解決他們的切身問題時,他們是選擇團結起來要求制度上的改革,還是把責任個人化各自去托關係找出路?如果是後者,又有沒有進一步加深社會不平等和無力感?

這些問題,得交給比我懂的人回答。但答問之前,我想無論在中國大陸或在香港,都離不開一個前設:要為回答留下空間。答問需要空間,因為任何權力本質上都不喜歡受到質疑,專制政權尤甚。答問是一種異議,因為專制只能容許他們認可的答案,不容其他解讀。專制審查外,也得認清自身局限。香港人經歷過太多次的自我欺騙,並因而損失許多光陰。現實是我們都是經驗有限的個體,面對越宏大的問題,特別是國家民族或者各種思想主義等的問題,不應輕易確定已有簡單圓滿的答案。

不停和其他人的回答對比,是保持謙卑和守護答問空間的好方法。我比較幸運,在寫上面十多萬字之前,已有大量前人留下來扎實的香港研究可供參考,讓我不用抽象瞎扯。

相對來說,過去數年的中國大陸,援引不同意見的空間卻越見收窄,社會變得一言堂。我擔心在這樣的環境下,大家都變得自信滿滿,中國會很容易會走歪路。我懷疑,今天中國面對逆轉的國際處境,持續改革開放遇到巨大阻力,反映中國過去數年確實走了歪路。而中港關係近年的急速惡化,只是這條歪路帶來的眾多問題的其中一例。

中港政治關係應該變成怎麼樣,我沒有很好的答案。但無論你對中港政治關係未來的希冀為何,覆巢之下、焉有完卵,香港沒有期望中國變得更保守的本錢。我不是說香港人能主動做什麼去改變中國。我甚至懷疑,香港能繼續存活下去已是一個很大的貢獻。因為香港的存在本身,對中國大陸也好,對台灣也好,都很寶貴。

我在中國大陸聽過不少聲稱香港人沒有國家視野的批評。我其實頗同意這觀點。在中國大陸長大的年輕人,不少會想像整個中國都是他們打拼的舞台,會問自己應該到何處發展,香港的年輕人則極少會有這種想法。但沒有國家視野,不一定是件壞事。

中國大陸近年流行說「國家在下一盤很大的棋」,香港不少親中權貴也喜歡指點香港人沒有大局觀念,不懂得從國家的立場來思考香港定位。不過,人和棋子,總有些分別。棋子只管執行任務,隨時準備犧牲,必要時棄車保帥。所以,棋子不能有思想。但人有思想,而且應該要有思想。就算人選擇要做棋子,要為集體利益付出,也得最少問一問下棋的那隻手是誰,為甚麼他有資格,而我又可不可以參與決定下一步棋如何走。這些問題,香港人喜歡問。

我接受現在在中國大陸問這些問題的人不多。當發展的速度快,大家都好像得到發展的好處時,很少人會介意那個集體其實是什麼。反正只要餅大了,自己那一份應該也會變大,何必質疑自己有沒有權力參與決策過程,特別是這種質疑會為自己帶來麻煩。大家甚至會開始為這個所謂的集體辯護,如同為自己辯護,心甘情願地去說項。至於個人對集體的付出應否永遠不計條件,誰有權決定甚麼才是集體的真正利益,不必深究。但總有一天,當發展的速度沒有那麼快,不能再讓大家都分得到好處,當制度性的不平等和不公義變得不能迴避時,一些艱難的價值觀問題就變得必須面對。

而這些,我懷疑就是這個「沒有國家視野」的香港在中國應扮演的角色:為答問一些讓人不舒服的問題,留下一點空間。當全國各省市區都在努力走同一條路的時候,香港對中國的意義就在於保留一個不同的可能。至於這個不同的可能最後用不用得著,合不合適,有誰知道?但留一扇窗,本身就有價值。

最後,我想借用中國研究的學界泰斗科大衛教授的一番話作結。除了在新聞與傳播學院任教社會分析,我也有在中文大學的中國研究中心任教中港關係,而科大衛教授正是我們的課程總監。記得有一年的開學禮,現場也有很多來自中國大陸的學生,科教授這樣說:「你們來自中國大陸,為什麼還要來香港學中國研究呢?我想我們和在中國大陸的同行有一個分別:香港有異議的傳統。你們來香港,只會留一年的時間。我不期望你會在一年之內完全改變,這也不應該發生,這會很可怕。但我希望我們可教曉你們問問題的重要。」

我想,這就是探討香港眾多問題的意義。香港很獨特,但香港不孤單。身分該如何理解?歷史該如何記憶?權力是否必須服從?制度該如何改變?面對系統性敗壞時能如何力抗?這些問題,不止香港人在面對,也不止香港人在嘗試回答。對香港的未來,我不敢盲目樂觀。但如果香港的故事能激發更多人勇於問問題,不急於下定論,慢慢去找答案,容許不同意見的存在,也算在世上留下了一個難能可貴的印記。



(《香港第一課》已連載完畢,在此感謝 Matters 團隊的支持和協助。)

(編者註:感謝梁啟智先生!)